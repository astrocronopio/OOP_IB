
%version 2: \usepackage{hyperref}


%%%%%%%%%%%%%%%%%%%%%%%%%%%%%%%%%%%%%%%%%%%%%%%%%%%%%%%%%%%%%%%%%%%%%%%%
%Para las ecuaciones siempre es Ec.(n).
%Para las figuras siempre es Fig.n, incluso en el caption de la figura. Tambien las Tablas
%Para las referencias es [n]
%%%%%%%%%%%%%%%%%%%%%%%%%%%%%%%%%%%%%%%%%%%%%%%%%%%%%%%%%%%%%%%%%%%%%%%%

\documentclass[
reprint,
%notitlepage,
% superscriptaddress,
%groupedaddress,
%unsortedaddress,
%runinaddress,
%frontmatterverbose, 
%preprint,
%showpacs,preprintnumbers,
%nofootinbib,
%nobibnotes,
%bibnotes,
% 10. pt,
amsmath,
amssymb,
%aps,
% pra,
% prb,
rmp,
%tightenlines %esto hizo el milagro de sacar los espacios en blancos estocásticos (?)
% prstab,
%prstper,
%floatfix,\textbf{}
]{revtex4-1} %Instalar primero para usarlo. Paquete malo.

%\documentclass[onecolumn, aps, amsmath,amssymb ]{article}
\usepackage{lipsum}  
\usepackage{graphicx}% Include figure files
\usepackage{subfig}
\usepackage{braket}
\usepackage{comment} %comment large chunks of text
\usepackage{dcolumn}% Align table columns on decimal point
\usepackage{bm}% bold math
%\usepackage{hyperref}% add hypertext capabilities
\usepackage[mathlines]{lineno}% Enable numbering of text and display math
%\linenumbers\relax % Commence numbering lines
\usepackage{mathtools} %% Para el supraíndice

\usepackage[nice]{nicefrac}

%%%%%%%El Señor Español%%%%%%%%%%%%%%%%%%%%%%%%%%%
\usepackage[utf8]{inputenc} %acento
\usepackage[
spanish, %El lenguaje.
es-tabla, %La tabla y no cuadro.
activeacute, %El acento.
es-nodecimaldot %Punto y no coma con separador de números
]{babel}
\usepackage{microtype} %para hacerlo más bonito :33 como vos (?) 
%%%%%%%%%%%%%%%%%%%%%%%%%%%%%%%%%%%%%%%%%%%%%%%%%%%
%%%%%%%%% Para que las imágenes se queden dónde las quiero (?
\usepackage{float}
%%%%%%%%%%
\usepackage{enumitem}
\usepackage{hyperref} % Para usar \url

%%%%%%%%Cambia a Fig de Figure%%%%%%%%%%
\makeatletter
\renewcommand{\fnum@figure}{Fig. \thefigure} 
\makeatother
%%%%%%%%%%%%%%%%%%%%%%%%%%%%%%%%%%%%%%%%
\raggedbottom

\begin{document}

\title{MQTT++}
\author{Evelyn~G.~Coronel}

\affiliation{
Programación Orientada a Objetos\\ Instituto Balseiro\\}

\date[]{\lowercase{\today}} %%lw para lw, [] sin date

\maketitle
% \onecolumngrid

\section{MQTT simplificado en \texttt{C++}}

En este trabajo se basó en la implementación del protocolo de MQTT utilizando los temas de la materia. 


\subsection{Librería \texttt{mqtt.hpp}}

Esta  librería están almacenadas las variables reservadas para el protocolo, dentro del \texttt{namespace mqtt}. También se implementó una clase que asigna valores de identificación (ID) únicos a los clientes.

\subsection{Librería \texttt{mqtt\_errors.hpp}}

\subsection{Librería \texttt{mqtt\_message.hpp}}

\subsection{Librería \texttt{mqtt\_client.hpp}}

    En esta librería están implementadas las clases utilizada para instanciar objetos que se comporten con un cliente. La misma está dentro del \texttt{namespace mqtt\_client}.

    \subsubsection{Clase \texttt{mqtt\_client::client\_virtual}}
    En esta clase esta implementada de tal forma que la misma y sus hijos no puedan copiarse. También se declaran funciones virtuales puras para que luego van a ser implementadas en las clases hijas.

    \subsubsection{Clase \texttt{mqtt\_client::client}}

    Esta clase es heredada de la clase \verb|mqtt_client::client_virtual|. Se asigna un ID al azar, usando \verb|mqtt_client::get_id()| y \verb|mqtt_client::set_id(<unsigned int>)|  cuando se conecta al broker que transmite los mensajes

    Un cliente puede estar conectado a un solo server ya que al conectarse/desconectarse la variable \verb|mqtt_client::Connected| cambia \verb|mqtt::CONNECTED| / \verb|mqtt::DISCONNECTED|. El estado del cliente se obtiene mediante la función \verb|mqtt_client::isConnected()|.

    Un cliente puede estar suscrito a varios topics al mismo tiempo, la función \verb|mqtt_client::client::subscribe(<string>)|  agrega un nuevo topic a una lista miembro de la clase \verb|mqtt_client::client|, la función \verb|mqtt_client::client::get_topic()| devuelve dicha lista.

    El cliente solo puede estar conectado a un servidor
    al mismo tiempo. Tiene la opción de hacer que sus mensajes siempre vayan por defecto, al inicio de la cola de mensajes con  \verb|mqtt_client::set_Priority(mqtt::HIGH)|.

    La función \verb|mqtt_client::client::reply()| debe ser implementada por cada tipo de cliente, ya  que cada uno va a tener una forma de responder distinta.


\subsection{Librería \texttt{mqtt\_server.hpp}}

\subsection{Librería \texttt{mqtt\_broker.hpp}}

La librería esta con el \texttt{namespace mqtt\_broker}
\subsubsection{Clase \texttt{mqtt\_broker}}
Esta clase es una clase hija de \verb|mqtt\_server::server|
Como atributos la clase tiene una lista con las direcciones de los clientes conectados al broker, un generador de IDs, además de un \verb|mutex| y un \verb|condition_variable| para manejar la lista de clientes y la cola de mensajes.

Los métodos de esta clase incluyen a:

\begin{itemize}
    \item[-] \verb|sin_subs()|:
    Devuelve \verb|False| si no hay clientes conectados, \verb|True| caso contrario.
    
    \item[-] \verb|publish_from(<cliente *, mensaje*>)| 
    
    Publica un cliente a los demás clientes mediante el servidor.
    
    \item[-] \verb|publish(<mensaje*>)|
    
    Publica el servidor a todos los clientes

    \item[-] \verb|connect(<cliente *, string>)| 
    
    Conecta un cliente al servidor
    
    \item[-] \verb|disconnect(<cliente *>)|  
    
    Desconecta un cliente del servidor.
    
    \item[-] \verb|broadcast_message()| 
    
    La función que publica los mensajes a todos los clientes.
\end{itemize}



\subsection{Ejemplos}


\end{document}